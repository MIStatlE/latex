\documentclass[10pt,openany]{book} % 可改 11pt 或 openright
\usepackage{miextras} % ← 使用上面的风格文件

\begin{document}

% 封面(无眉脚)
\MakeBookCoverUltraSimple{集中不等式(示例书名)}{高维概率与机器学习理论中的浓缩现象}{上册 · 预印本}

% 目录(整段无眉脚)
\setcounter{tocdepth}{2}
\TOCwithoutHeadFoot

% 主体
\mainmatter

\part{预备知识与工具}
\chapter{概率预备(示例)}
\ChapterEpigraph[Kolmogorov]{The theory of probability as a mathematical discipline can
and should be developed from axioms in exactly the same way as geometry and algebra.}

\begin{SideBar}
\textbf{阅读建议}:本章覆盖后续章节频繁使用的定义与工具。
\end{SideBar}

\section{Orlicz 范数与次高斯}
\begin{definition}[Orlicz 范数]
随机变量 $Z$ 的 $\psi_2$ 范数定义为
\[
\|Z\|_{\psi_2} := \inf\{s>0:\ \E e^{Z^2/s^2}\le 2\}.
\]
\end{definition}

\begin{lemma}[基本性质(示例)]
若 $X$ 次高斯,则 $X^2-\E X^2$ 次指数。
\end{lemma}

\begin{KeyBox}
\textbf{要点}:\(\psi_2\)-有界 $\Rightarrow$ 指数尾界;平方中心化 $\Rightarrow$ \(\psi_1\)-有界。
\end{KeyBox}

\section{Bernstein 不等式}
\begin{theorem}[Bernstein(示例)]
设 $Z_i$ 独立、$\|Z_i\|_{\psi_1}\le v$,则……
\end{theorem}

\begin{Takeaway}
小偏差 $\sim \|a\|_2$,大偏差 $\sim \|a\|_\infty$。
\end{Takeaway}

\part{主结果}
\chapter{Hanson--Wright 不等式(示例)}
\section{定理与讨论}
\begin{Example}
这是一个示例环境,用于给出具体计算或反例。
\end{Example}

% 附录
\appendix
\part{附录}
\chapter{符号与记号(示例)}
\begin{itemize}
  \item \(\E\):数学期望
  \item \(\Var\):方差算子
\end{itemize}

% 参考文献
\backmatter
\begin{thebibliography}{9}
\bibitem{vershynin}
Vershynin, R. \emph{High-Dimensional Probability}. CUP, 2018.
\end{thebibliography}

\end{document}